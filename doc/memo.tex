\documentclass[xelatex,ja=standard]{bxjsarticle}
\setCJKmainfont[BoldFont=NotoSerifCJKjp-Black]{NotoSerifCJKjp-Light}
\setCJKsansfont[BoldFont=NotoSansCJKjp-Black]{NotoSansCJKjp-Light}
\setCJKmonofont[BoldFont=NotoSansCJKjp-Black]{NotoSansCJKjp-Light}
\usepackage{physics}
\usepackage{bm}
\usepackage[colorlinks=true, linkcolor=black, citecolor=black, urlcolor=blue]{hyperref}



\title{シュレーディンガー方程式を有限要素法で解く}
\author{山本峻介}
\date{\today}

\begin{document}

\maketitle

\section{水素原子のシュレーディンガー方程式}
\subsection{動経成分波動関数}
水素原子のハミルトニアンは
\begin{align}
    i\hbar\frac{\partial}{\partial t}\psi(\bm{r},t) = -\frac{\hbar^2}{2m}\nabla^2\psi(\bm{r},t) -\frac{e^2}{r}\psi(\bm{r},t)
\end{align}
である。ここで、$\psi(\bm{r},t)$は波動関数、$\hbar$はディラック定数、$m$は電子の質量、$e$は電気素量、$\bm{r}$は位置ベクトル、$t$は時間である。
エネルギー$E$の固有関数$\psi(\bm{r})$は
\begin{align}
    -\frac{\hbar^2}{2m}\nabla^2\psi(\bm{r}) -\frac{e^2}{r}\psi(\bm{r}) = E\psi(\bm{r})
\end{align}
動経成分は
\begin{align}
    -\frac{\hbar^2}{2m}\qty(\frac{1}{r^2}\dv{r}r^2\dv{r} - \frac{l(l+1)}{r})R(r)
     - \frac{e^2}{r}R(r) = ER(r)
\end{align}
となる。$\chi = rR$を導入すると、
\begin{align}
    -\frac{\hbar^2}{2m}\qty(\dv[2]{r} - \frac{l(l+1)}{r^2})\chi - \frac{e^2}{r}\chi = E\chi
\end{align}
となる。

\subsection{弱形式}
固有値方程式
\begin{align}
    \dv[2]{x}f + \frac{a}{x^2}f + \frac{b}{x}f = \lambda f 
\end{align}
に対して、弱形式を求める。$f$に対してテスト関数$g$を掛けて積分すると、
\begin{align}
    \int_0^Lg\qty(\dv[2]{x}f + \frac{a}{x^2}f + \frac{b}{x}f)\dd{x} &= \lambda\int_0^Lgf\dd{x}\notag\\
    \qty[g\dv{f}{x}]_0^L+\int_0^L\qty(\dv{g}{x}\dv{f}{x} + \frac{a}{x^2}gf + \frac{b}{x}gf)\dd{x}
     &= \lambda\int_0^Lgf\dd{x}
\end{align}
境界条件$\dv{f}{x}|_{x=0} = \dv{f}{x}|_{x=L} = 0$を課すと、
\begin{align}
    \int_0^L\qty(\dv{g}{x}\dv{f}{x} + \frac{a}{x^2}gf + \frac{b}{x}gf)\dd{x}
     &= \lambda\int_0^Lgf\dd{x}\label{eq:weak}
\end{align}
となる。
\subsection{有限要素法}
(\ref{eq:weak})の被積分関数は、$\dv{g}{x}\dv{f}{x}$と$h(x)g(x)f(x)$からなる。
区間$\qty[0,L]$を$N$等分する。$e$番目の区間を$[x^e_0,x^e_1]$とし、$g_{0(1)} = g(x=x^e_{0(1)})$, $f_{0(1)} = f(x=x^e_{0(1)})$, $l^e = x^e_1 - x^e_0$とすると、
\begin{align}
    g(x) &= \frac{x^e_1 - x}{l^e}g_0 + \frac{x-x^e_0}{l^e}g_1,\\
    f(x) &= \frac{x^e_1 - x}{l^e}f_0 + \frac{x-x^e_0}{l^e}f_1,\\
    h(x) &= \frac{x^e_1 - x}{l^e}h_0 + \frac{x-x^e_0}{l^e}h_1
\end{align}
となる。これらにより、
\begin{align}
    \int_0^L \dv{g}{x}\dv{f}{x}\dd{x} 
    &= \sum_{e=1}^N \frac{1}{l^e_2}\int_{0_x^e}^{x^e_1}\qty(-g^e_0 + g^e_1)\qty(-f^e_0 + f^e_1)\dd{x}\notag\\
    &= \sum_{e=1}^N \frac{1}{l^e_2}\int_{x^e_0}^{x^e_1}\mqty(g^e_0 && g^e_1)\mqty(-1\\1)\mqty(-1 && 1)\mqty(f^e_0 \\ f^e_1)\dd{x}\notag\\
    &= \sum_{e=1}^N \frac{1}{l^e_2}\int_{x^e_0}^{x^e_1}\mqty(g^e_0 && g^e_1)\mqty(1 && -1 \\ -1 && 1)\mqty(f^e_0 \\ f^e_1)\dd{x}\notag\\
    &= \sum_{e=1}^N \frac{1}{l^e}\mqty(g^e_0 && g^e_1)\mqty(1 && -1 \\ -1 && 1)\mqty(f^e_0 \\ f^e_1),
\end{align}
\begin{align}
    \int_0^L hgf\dd{x}
    =& \sum_{e=1}^N \frac{1}{l^e_3}\int_{x^e_0}^{x^e_1}
    \qty(\qty(x^e_1 - x)h^e_0 + \qty(x-x^e_0)h^e_1)
    \qty(\qty(x^e_1 - x)g^e_0 + \qty(x-x^e_0)g^e_1)\notag\\
    &\times\qty(\qty(x^e_1 - x)f^e_0 + \qty(x-x^e_0)f^e_1)
    \dd{x}\notag\\
    =& \sum_{e=1}^N \frac{1}{l^e_3}\int_{x^e_0}^{x^e_1}
    \mqty(g^e_0 && g^e_1) \mqty(x^e_1 - x \\ x-x^e_0)
    \qty(\qty(x^e_1 - x)h^e_0 + \qty(x-x^e_0)h^e_1)\notag\\
    &\times\mqty(x^e_1 - x && x-x^e_0)\mqty(f^e_0 \\ f^e_1)
    \dd{x}\notag\\
    =& \frac{1}{12}\sum_{e=1}^N l^e
    \mqty(g^e_0 && g^e_1) \mqty(3h^0 +h^1 && h^0 +h^1\\ h^0 +h^1&&h^0 +3h^1)\mqty(f^e_0 \\ f^e_1)
\end{align}
となる。
\begin{align}
    V_\mathrm{cf} = \frac{a}{x^2},\quad V_\mathrm{c} = \frac{b}{x}
\end{align}
と定義すると、(\ref{eq:weak})は
\begin{multline}
    \sum_{e=1}^N \mqty(g^e_0 && g^e_1)\bigg[\frac{1}{l^e}\mqty(1 && -1 \\ -1 && 1)
     + \frac{l^e}{12}\mqty(3V_{\mathrm{cf}\,0} +V_{\mathrm{cf}\, 1} && V_{\mathrm{cf}\, 0} +V_{\mathrm{cf}\, 1}\\ V_{\mathrm{cf}\, 0} +V_{\mathrm{cf}\, 1}&&V_{\mathrm{cf}\, 0} +3V_{\mathrm{cf}\, 1})\\
     + \frac{l^e}{12}\mqty(3V_{\mathrm{c}\,0} +V_{\mathrm{c}\, 1} && V_{\mathrm{c}\, 0} +V_{\mathrm{c}\, 1}\\ V_{\mathrm{c}\, 0} +V_{\mathrm{c}\, 1}&&V_{\mathrm{c}\, 0} +3V_{\mathrm{c}\, 1})\bigg]\mqty(f^e_0 \\ f^e_1)=\lambda\sum_{e=1}^N \mqty(g^e_0 && g^e_1)\mqty(f^e_0 \\ f^e_1)
\end{multline}
\end{document}
